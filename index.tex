% Dokumentanklasse: a4paper, 14pt
% Beschreibung:     Dokumentenformat
% Option:           extraarticle - ?
\documentclass[a4paper,14 pt]{extarticle}

% Paket:            a4paper
% Beschreibung:     A4 Seitenabstände
% Option:           geometry
\usepackage{geometry}
\geometry{
	paper=a4paper, 	% Papierformat
	%top=3cm, 		% Kopf-Spannweite
	%bottom=1.5cm,	% Fuß-Spannweite
	%left=4.5cm, 	% Linke-Spannweite
	%right=4.5cm,	% Rechte-Spannweite
	%showframe,		% Uncomment to show how the type block is set on the page
}

% Paket:            ansmath
% Beschreibung:     Zum darstellen von mathematischen Formeln
\usepackage{amsmath}

% Paket:            ngerman
% Beschreibung:     Deutsche Rechtschreibung
% Option:           babel - Sibentrennung
%\usepackage{ngerman}
\usepackage[ngerman]{babel}

% Paket:            utf8
% Beschreibung:     Stellt Umlaute richtig dar
% Option:           inputenc - Erlaubt die Darstellung der gleichen Zeichen (Character) wie sie in strin überliefert werden
\usepackage[utf8]{inputenc}

% Paket:            makeindex
% Beschreibung:     Ermöglicht das Indexieren von Wörter und den Befehl \printindex um den Index auszugeben
\usepackage{makeidx}
\makeindex

% Paket:            natbib
% Beschreibung:     Für Zitate
% Option:           round - ?
%\usepackage[round]{natbib}

% Paket:            fancyhdr
% Beschreibung:     Ermöglich ein generelles Seitenlayout ein zu stellen mit Kopf und Fußzeile.
\usepackage{fancyhdr}

% Paket:            graphicx
% Beschreibung:     Einbinden von Bildern
% Option:
\usepackage{graphicx}

% Paket:            enumitem
% Beschreibung:     Zeilenabstände bei Aufzählungen definieren
% Option:
\usepackage{enumitem}

% Paket:            float
% Beschreibung:     Zum Ausrichten von Tabellen und Spalten bzw deren zentrierung
% Option:
% Restriktion:      Muss von Paket hyperref geladen werden. Ansonsten funktioniert das Paket nicht.
\usepackage{float}

% Paket:            appendix
% Beschreibung:     Das Paket dient dazu, ausschließlich das Thema einer Überschrift in das Inhaltsverzeichnis zu überführen
% Option:           appendix - Überführt die Überschriften des Anhangs richtig ins das Inhaltsverzeichnis
\usepackage[titletoc]{appendix}

% Paket:            setspace
% Beschreibung:     Setz über die optionen den Zeilenabstand
% Optionen:         Möglicher Zeilenabstand
%                   singlespacing = 1,0
%                   onehalfspacing = 1,5
%                   doublespacing = 2,0
% Restriktion:      Muss von Paket hyperref geladen werden. Ansonsten funktioniert das Paket nicht.
\usepackage[onehalfspacing]{setspace}

% Packet:           Hyperref
% Beschreibung:     Importiert hyperref um Querverweise mittels \hyperref zu erzeugen.
\usepackage[]{hyperref}
\hypersetup{
  pdftitle={PDF-Title},
  pdfauthor={Markus Pesch},
  pdfsubject={PDF-Subject}
}

% Packet:           Minted
% Beschreibung:     Dient zum highlining von Quellcode wie beispielsweise Java, Bash oder Python.
\usepackage{minted}
\usemintedstyle{emacs}

% Packet:           tabularx
% Beschreibung:     Werden Tabellen mit diesem Paket erstellt, ist es möglich Zeilenumbrüche in einer Zelle zu erzeugen
\usepackage{tabularx}



% Paket:            biblatex
\usepackage[
  style=authoryear-icomp,  % Zitierstil
  isbn=false,              % ISBN nicht anzeigen, gleiches geht mit nahezu allen anderen Feldern
  pagetracker=true,        % ebd. bei wiederholten Angaben (false=ausgeschaltet, page=Seite, spread=Doppelseite, true=automatisch)
  maxbibnames=50,          % maximale Namen, die im Literaturverzeichnis angezeigt werden (ich wollte alle)
  maxcitenames=3,          % maximale Namen, die im Text angezeigt werden, ab 4 wird u.a. nach den ersten Autor angezeigt
  autocite=inline,         % regelt Aussehen für \autocite (inline=\parancite)
  block=space,             % kleiner horizontaler Platz zwischen den Feldern
  backref=true,            % Seiten anzeigen, auf denen die Referenz vorkommt
  backrefstyle=three+,     % fasst Seiten zusammen, z.B. S. 2f, 6ff, 7-10
  date=short,              % Datumsformat
  backend=biber
]{biblatex}
\setlength{\bibitemsep}{1em}     % Abstand zwischen den Literaturangaben
\setlength{\bibhang}{2em}        % Einzug nach jeweils erster Zeile

\addbibresource{literatur//bibliothek.bib}


% Packet:           glossaries
% Beschreibung:     Glossar einbinden und Glossarbefehle bereitstellen
% Option:           Gebe Glossar auch als section im Inhaltsverzeichnis aus
\usepackage[toc,section=section]{glossaries}
\makeglossaries
\include{glossar//glossar}

% Einstellungen überschreiben
\newcommand{\myparagraph}[1]{\paragraph{#1}\mbox{}\\}

% Start des Dokuments
\begin{document}

  % Fetch Commit ID and Date
  \immediate\write18{./git-info.sh commit > git-id.tmp}
  \immediate\write18{./git-info.sh date > git-date.tmp}
  \immediate\write18{./git-info.sh url > git-url.tmp}

  % Importiere weitere .tex Dokumente
  \begin{titlepage}
	\begin{center}
		
		\begin{huge}
			\begin{singlespace}
				\textbf{Seminar - Data Science}
			\end{singlespace}
		\end{huge}
		
		\vspace{1.2cm}
		
		\begin{Large}
			Das Revisionssystem git
		\end{Large}
		
		\vspace{0.5cm}
		
		\begin{figure}[h]
			\centering
			\includegraphics[width=0.85\textwidth]{images//logo.png}
			\label{img:fh-trier-logo}
		\end{figure}
		
		\vspace{1.5cm}
		
		\begin{large}
			Markus Pesch\\
			peschm@hochschule-trier.de
		\end{large}
		
		\vspace{2cm}
		
		Latex Quellcode auf \input{./git-url.tmp} \\
		basierend auf git commit \input{git-id.tmp} vom \input{git-date.tmp}
		
	\end{center}
\end{titlepage}

  \pagebreak

  % Pagestyle
  % Setze das Seitenlayout auf leer um Fuß und Kopfzeilen zu unterdrücken
  \pagestyle{empty}

  % Importiere weitere .tex Dokumente
  %\section*{Vorwort}
Dieses Dokument soll den Leser den Umgang mit Git näher bringen. Dazu wird Ihm der Umgang mit den gängigen Git-Befehlen näher gebracht und der Workflow von üblichen Online-Repositories erläutert. 

Als Beispiel wird in diesem Dokument ein Workflow eingerichtet, die dem Entwickler ermöglicht Quellcodeverbesserungen an diesen Dokument durch Merge Requests vorzunehmen.

\newpage


  \include{agenda//agenda}

  % Pagestyle
  % Setze das Seitenlayout auf fancyhdr um Fuß- und Kopfzeilen zu setzen
  \pagestyle{fancy}

  % Löscht alle Kopf- und Fußzeilen des pagestyles fancyhdr
  \fancyhf{}

  % Fuß- und Kopfzeile des Paketes fancyhdr
  % [L] - Linkeseite      [O] - Ungerade Seitenzahlen         [LE,LO] - Linkeseite, Gerade- und Ungerade Seitenanzahlen
  % [C] - Mitte           [E] - Gerade Seitenanzahlen         [CE]    - Seitenmitte, nur gerade Seitenanzahlen
  % [R] - Rechteseite                                         [RO]    - Rechteseite, nur ungerade Seitenanzahlen
  % \fancyhead    Kopfzeile
  % \fancyfoot    Fußzeile
  \fancyhead[L]{\rightmark}
  \fancyhead[R]{Seite \thepage}

  % Pixelstärke der Kopfzeilenlinie
  \renewcommand{\headrulewidth}{1pt}

  % Setze die Seitenbeginn zurück
  \setcounter{page}{1}

  % Importiere weitere .tex Dokumente
  \label{sec:einfuehrung}
\section{Einführung}
Dieses Kapitel bietet eine Einführung in die Grundbegriffe von Git und deren Konfigurationseinstellungen. Fortlaufend wird auf Basis dieses \LaTeX{} Dokuments gezeigt, wie ein Projekt mit Git unter Versionskontrolle gestellt werden kann. Dazu werden die wichtigsten Git-Kommandos näher erläutert.

\label{sec:einfuehrung.grundbegriffe}
\subsection{Grundbegriffe}
Um dem Umgang mit Git zu erläutern ist es notwendig, einige Fachbegriffe zu verstehen. 

\myparagraph{Versionskontrollsystem} 
Ein Versionskontrollsystem dient zur Verwaltung und Versionierung von Software. Bekannte Projekte die unter dem Versionskontrollsystem Git stehen sind beispielsweise der Linux-Kernel und Git selbst. Beide Projekte stehen unter der Lizenz GPL. Der Quellcode kann mittels git bezogen werden. Dazu später mehr.

Bei Versionskontrollsystemen wird unterschieden zwischen zentralen und dezentralen Systemen. Git ist ein dezentrales System. Der Entwickler ist nicht abhängig von dem Server auf dem die Versionshierarchie gespeichert ist. Der Entwickler ist in der Lage mit Git die komplette Versionshierarchie auf sein eigenes System zu klonen und auf dieser Basis zu arbeiten.   

\myparagraph{Repository}
Git  verwaltet von jeder Datei unterschiedliche Zustände bzw. Versionen. Diese werden in dem Repository gespeichert. Mithilfe des Repositories ist es möglich, auf jeden Zustand einer Datei zurück zu springen.

\myparagraph{Working Tree}
Alle Modifikationen an Dateien bzw. dem Quellcode werden im Working Tree vorgenommen. Andere Bezeichnungen hierfür sind auch Working Directory oder Workspace. Wobei bei vielen IDEs im Heimatverzeichnis das Verzeichnis workspace erstellt wird um Projekte zu speichern und unter Versionskontrolle zu setzen. 

\myparagraph{Commit}
Ein Commit nimmt jede Veränderung, Modifikation oder Erstellung an Dateien im Working Tree auf. Zusätzlich speichert der Commit noch weitere Informationen wie den Benutzernamen und dessen E-Mail mit Datum und optional gpg-Signatur ab.  

\myparagraph{HEAD}
HEAD bildet eine Referenz ab, in welchem Zustand der Entwickler sein Working Tree vor findet. 

\myparagraph{Branch}
Jede Software kann in ihrem Verlauf mehrere Entwicklungsverzweigungen haben um beispielsweise in jeder Verzweigung einzelne Teilaufgaben wie Patches oder neuen Features von Software zu entwickeln. Diese Verzweigungen werden Branches genannt. Jede Verzweigung kann später durch einen Merge oder Rebase wieder zurück geführt werden.


\label{sec:einfuehrung.git}
\subsection{Einrichtung von Git}
Wie bereits in in Kapitel \ref{sec:einfuehrung.grundbegriffe} erklärt, speichert Git bei jedem Commit den Benutzernamen und E-Mail ab. Bevor nun dieses Dokument, das als \LaTeX{} Projekt angelegt wurde, durch Git verwaltet wird, ist es Sinnvoll den Benutzernamen und die E-Mail Adresse anzupassen \footcite{git-1.6-your-identity}.

\begin{minted}[linenos, framesep=2mm, fontsize=\small]{bash}
$ git config --global user.name "Hugo McKinnock"
$ git config --global user.email "hugo.mckinnock@example.com"
\end{minted}

Ist ein gpg Schlüssel vorhanden, kann dieser direkt hinterlegt werden um Commits direkt im Hintergrund zu signieren. Dazu muss zuerst der gpg Schlüssel ausfindig gemacht werden und anschließend Git mitgeteilt werden.

\begin{minted}[linenos, framesep=2mm, fontsize=\small]{bash}
$ gpg --list-secret-keys --keyid-format 0xSHORT
/home/hugo/.config/gnupg/pubring.gpg
--------------------------------------
sec   rsa4096/0x85ED78DE 2018-03-15 [SC] [verfällt: 2018-09-15]
uid        [  ultimativ] Hugo McKinnock <hugo.mckinnock@example.com>
ssb   rsa4096/0x848DEAC1 2018-03-15 [E] [verfällt: 2018-09-15]

$ git config --global user.signingkey "0x85ED78DE"
$ git config --global commit.gpgSign true
\end{minted}

\section{Projekt auschecken}
Wir erstellen im Heimatverzeichnis, wie von den meisten IDE's bevorzugt, das Verzeichnis \textit{workspace} und klonen von einem Online-Repository dieses Projekt. Anschließend befindet sich im Verzeichnis \textit{workspace} der Unterordner \textit{sdsm\_ss17\_git} in den wir navigieren. 

\begin{minted}[linenos, framesep=2mm, fontsize=\small]{bash}
$ mkdir ~/workspace
$ git clone https://git.cryptic.systems/fh-trier/sdsm_ss17_git.git
$ cd sdsm_ss17_git
\end{minted}

\begin{INFO}
  Zum klonen eines Online-Repositories wird auf vielen Online-Plattformen wie \href{https://github.com}{GitHub} oder \href{https://gitlab.com}{GitLab} eine Adresse per \textit{HTTPS-} oder \textit{SSH-}Protokoll angeboten. 
  
  Für eine Verbindung per SSH-Protokoll muss ein Account auf diesen Online-Plattformen registriert sein und der öffentliche Schlüssel des asymmetrischen Verschlüsselungsverfahren hinterlegt sein.  
\end{INFO}

Das heruntergeladene Repository spiegelt nun den Zustand ab, auf dem sich die Referenz \textit{HEAD} befindet. Um dies zu überprüfen, bietet Git den Befehl \textit{git log} an. Wir lassen uns die neusten zwei Commits anzeigen.

\begin{minted}[linenos, framesep=2mm, fontsize=\small]{bash}
$ git log -2
commit 4660915d6c031598c77b49c6275e426f7c3a85c3 (HEAD -> master, 
origin/master, origin/HEAD)
Author: Markus Pesch <markus.pesch@cryptic.systems>
Date:   Thu Mar 15 16:49:12 2018 +0100

add: chapter 1

commit 1ca84c7271f3f3c306b779256bd1902497f44fb9
Author: Markus Pesch <markus.pesch@cryptic.systems>
Date:   Thu Mar 15 13:26:02 2018 +0100

fix: geometry
\end{minted}

Zu erkennen ist, dass sich die Referenz \textit{HEAD} auf der gleichen Position wie der Branch \textit{master} befindet. Der Branch \textit{master} ist ein Branch auf dem lokalen System des Entwicklers. Die Referenz \textit{origin/master} zeigt den Zustand an, auf dem das Online-Repository, von dem geklont wurde, steht.


  % Glossar
  \printglossaries
  \newpage

  % Abbildungsverzeichnis
  \listoffigures
  \newpage

  % Literaturverzeichnis
  \printbibliography

  % Erklärung
  %\include{erklaerung//erklaerung}

\end{document}

